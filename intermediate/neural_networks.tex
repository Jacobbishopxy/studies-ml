% @file:		neural_networds.tex
% @author:	Jacob Xie
% @date:		2023/03/18 11:57:29 Saturday
% @brief:

\documentclass[../studies-ml.tex]{subfiles}

\begin{document}

\subsection{神经元模型}

\begin{center}
  \begin{tabular}{ |p{3cm}||p{4cm}|p{6cm}|  }
    \hline
    \multicolumn{3}{|c|}{\textbf{神经元模型}}                                               \\
    \hline
    名称   & 英文                  & 描述                                                    \\
    \hline
    神经网络 & neural networks     & 由具有适应性的简单单元组成的广泛并行互联的网络,它的组织能够模拟生物神经系统对真实世界物体所作出的交互反应 \\
    \hline
    神经元  & neuron              & 神经网络中最基本的成分                                           \\
    \hline
    阈值   & threshold           & 公式中记作 $\theta$                                        \\
    \hline
    连接   & connection          &                                                       \\
    \hline
    激活函数 & activation function & 处理以产生神经元的输出                                           \\
    \hline
    挤压函数 & squashing function  &                                                       \\
    \hline
  \end{tabular}
\end{center}

\subsection{感知机与多层网络}

感知机(Perceptron)由两层神经元组成。如图示,输入层接受外界输入信号后传递给输出层,输出层是 M-P 神经元,也称阈值逻辑单元
(threshold logic unit),其中 $y = f(\sum_i w_i x_i - \theta)$,而 $f$ 为激活函数。

\begin{center}
  \begin{tikzpicture}[shorten >=1pt,->]
    \tikzstyle{unit}=[draw,shape=circle,minimum size=1]

    \node[unit](p) at (0,1){};
    \node[unit](x1) at (-0.8,0){};
    \node[unit](x2) at (0.8,0){};
    \node[below] at (x1.south){$x_1$};
    \node[below] at (x2.south){$x_2$};
    \node[below,left=0.35,yshift=-8] at (p.south){$w_1$};
    \node[below,right=0.35,yshift=-8] at (p.south){$w_2$};

    \draw [->] (x1) -- (p);
    \draw [->] (x2) -- (p);
    \draw (p) -- ++(0,1) node[xshift=-10]{$y$};

    \node [align=center, right=1.6] at (p) {\footnotesize{输出层}};
    \node [align=center, right=0.8] at (x2) {\footnotesize{输入层}};
  \end{tikzpicture}
\end{center}

一般而言,给定训练数据集,权重 $w_i\, (i=1,2,\dots,n)$ 以及阈值 $\theta$ 可通过学习得到。而阈值 $\theta$ 可视为一个固定输入
为 -1.0 的哑结点(dummy node)所对应的链接权重 $w_{n+1}$,因此权重和阈值的学习就可以统一为权重的学习。感知机的学习规则非常简单,
对训练样例 $(\pmb{x},y)$ 而言,如果当前感知机的输出位 $\hat{y}$,则感知机权重将这样调整:

\begin{equation}
  w_i \leftarrow w_i + \Delta w_i
\end{equation}

\begin{equation}
  \Delta w_i = \eta (y - \hat{y}) x_i
\end{equation}

其中 $\eta \in (0,1)$ 称为学习率(learning rate)。公式 5.2 为感知机学习算法中的参数更新式。

\begin{enumerate}[I]
  \item \textbf{感知机模型}

        感知机模型的式可表示为:

        \begin{align*}
          \begin{split}
            y & = f \Biggl( \sum_{i=1}^{n} w_i x_i - \theta \Biggr) \\
            & = f(\pmb{w}^T \pmb{x} - \theta)
          \end{split}
        \end{align*}

        其中,$x \in \mathbb{R}^n$ 即样本的特征向量,是感知机模型的输入;$\pmb{w}, \theta$ 是感知机模型的参数,权重 $w \in \mathbb{R}^n$,
        $\theta$ 为阈值。假定 $f$ 为阶跃函数,那么感知机模型的式可以表示为:

        \begin{equation*}
          y = \varepsilon (\pmb{w}^T \pmb{x} - \theta) = \begin{cases}
            1,\quad \pmb{w}^T \pmb{x} - \theta \ge 0; \\
            0,\quad \pmb{w}^T \pmb{x} - \theta < 0.
          \end{cases}
        \end{equation*}

        由于 $n$ 维空间中的超平面方程为
        \[
          w_1 x_1 + w_2 x_2 + \cdots + w_n x_n + b = \pmb{w}^T \pmb{x} + b = 0
        \]

        因此感知机模型式中的 $\pmb{w}^T \pmb{x} - \theta$ 可视为 $n$ 维空间中的一个超平面,将 $n$ 维空间划分为 $\pmb{w}^T \pmb{x} - \theta \ge 0$
        与 $\pmb{w}^T \pmb{x} - \theta < 0$ 的两个子空间(\textit{试想一下三维空间下的一个平面将空间切分为两部分})。
        那么落在前一个子空间的样本对应的模型输出值为 1,而后者为 0,如此实现了分类功能。

  \item \textbf{学习策略}

        给定一个线性可分的数据集 $T$,感知机的学习目标是求得能对数据集 $T$ 中的正负样本完全正确划分的分离超平面
        \[ \pmb{w}^T \pmb{x} - \theta = 0 \]

        假设此时误分类样本集合为 $M \subset T$,对任意一个误分类样本 $(\pmb{x},y) \in M$ 而言,当 $\pmb{w}^T \pmb{x} - \theta \ge 0$ 时,
        模型输出值为 $\hat{y} = 1$,样本真实标记为 $y = 0$;反之亦然。综上,以下式恒成立:
        \[ (\hat{y} - y)(\pmb{w}^T \pmb{x} - \theta) \ge 0 \]

        因此对于给定数据集 $T$,其损失函数可以定义为
        \[ L(\pmb{w}, \theta) = \sum_{\pmb{x} \in M} (\hat{y} - y)(\pmb{w}^T \pmb{x} - \theta) \]

        非负之和显然非负,因此损失函数为非负。当没有误分类点时,损失函数的值为 $0$;误分类点越少,误分类点离超平面越近,损失函数值就越小。
        因此对于给定的数据集 $T$,损失函数 $L(\pmb{w}, \theta)$ 是关于 $\pmb{w}, \theta$ 的连续可导函数(注意是关于 $\pmb{w}, \theta$
        的可导,意味着之后将要对其进行梯度下降算法,即对其使用导数计算)。

        连续:
        \[
          \lim_{x \to x_0} f(x) = f(x_0)
        \]

        可导:
        \[
          f'(x_0) = \lim_{\Delta x \to 0} \frac{\Delta y}{\Delta x} =
          \lim_{\Delta x \to 0} \frac{f(x_0 + \Delta x) - f(x_0)}{\Delta x}
        \]

  \item \textbf{学习算法}

        感知机模型的学习问题可以转化为求解损失函数的最优化问题。给定数据集
        \[ T = \{(\pmb{x}_1,y_1),(\pmb{x}_2,y_2),\cdots,(\pmb{x}_N,y_N),\} \]
        其中 $\pmb{x}_i \in \mathbb{R}, y_i \in \{0,1\}$,求参数 $\pmb{w}, \theta$ 使得损失函数最小化:
        \[
          \min_{\pmb{w},\theta} L(\pmb{w},\theta) =
          \min_{\pmb{w},\theta} \sum_{\pmb{x}_i \in M} (\hat{y}_i - y_i)(\pmb{w}^T \pmb{x} - \theta)
        \]

        其中 $M \subset T$ 为误分类样本集合。若将阈值 $\theta$ 视为一个固定输入为 $-1$ 的 “哑结点”(前文有提到),即:
        \[ -\theta = -1 \cdot w_{n+1} = x_{n+1} \cdot w_{n+1} \]

        那么 $\pmb{w}^T \pmb{x} - \theta$ 可简化为

        \begin{align*}
          \begin{split}
            \pmb{w}^T \pmb{x} - \theta & = \sum_{j=1}^{n} w_j x_j + x_{n+1} \cdot w_{n+1} \\
            &= \sum_{j=1}^{n+1} w_j x_j \\
            &= \pmb{w}^T \pmb{x_i}
          \end{split}
        \end{align*}

        其中 $\pmb{x_i} \in \mathbb{R}^{n+1},\, \pmb{w} \in \mathbb{R}^{n+1}$,有此可将最小化问题进一步简化:
        \[
          \min_{\pmb{w}} L(\pmb{w}) = \min_{\pmb{w}} \sum_{\pmb{x_i} \in M} (\hat{y}_i - y_i) \pmb{w}^T \pmb{x_i}
        \]

        假设误分类样本集合 $M$ 固定,那么可以求得损失函数 $L(\pmb{w})$ 的梯度
        \[ \nabla_{\pmb{w}} L(\pmb{w}) = \sum_{\pmb{x_i} \in M} (\hat{y}_i - y_i) \pmb{x_i} \]

        感知机的学习算法具体采用的是\textbf{随机梯度下降法},即在最小化的过程中,不是一次使 $M$ 中所有误分类点的梯度下降,
        而是一次随机选取一个误分类点,并使其梯度下降。所以权重 $\pmb{w}$ 的更新式为
        \begin{gather*}
          \pmb{w} \leftarrow \pmb{w} + \Delta \pmb{w}, \\
          \Delta \pmb{w} = -\eta (\hat{y}_i - y_i) \pmb{w} = \eta (y_i - \hat{y}_i) \pmb{w}
        \end{gather*}

        即 $\pmb{w}$ 中的某个分量 $w_i$ 的更新式即式 5.2。

        备注:这里随机的意义在于每次调整的权重 $w$ 不会对某个/些样本点产生依赖,即常说的“过拟合”。
\end{enumerate}

本章其余名词:

\begin{center}
  \begin{tabular}{ |p{4cm}||p{7cm}|  }
    \hline
    \multicolumn{2}{|c|}{\textbf{神经元模型}}               \\
    \hline
    名称       & 英文                                      \\
    \hline
    功能神经元    & functional neuron                       \\
    \hline
    线性可分     & linearly separable                      \\
    \hline
    收敛       & converge                                \\
    \hline
    震荡       & fluctuation                             \\
    \hline
    隐层或隐含层   & hidden layer                            \\
    \hline
    多层前馈神经网络 & multi-layer feedforward neural networks \\
    \hline
    连接权      & connection weight                       \\
    \hline
  \end{tabular}
\end{center}

\newpage

\subsection{误差逆传播算法}

误差逆传播算法(error BackPropagation,简称 BP)。

给定训练集 $D = \{(\pmb{x}_1, \pmb{y}_1), (\pmb{x}_2, \pmb{y}_2),\dots,(\pmb{x}_m, \pmb{y}_m)\},\,
  \pmb{x}_i \in \mathbb{R}^d,\, \pmb{y}_i \in \mathbb{R}^l$,即 input 为 $d$ 维,output $l$ 维。如图:

\begin{center}
  \tikzset{%
    every neuron/.style={
        circle,
        draw,
        minimum size=0.7cm
      },
    neuron missing/.style={
        draw=none,
        scale=2,
        text width=0.444cm,
        execute at begin node=\color{black}$\hdots$
      },
  }

  \begin{tikzpicture}[x=1cm, y=1.5cm, >=stealth]

    % ================================================================================================
    % 神经元
    % ================================================================================================
    \foreach \m [count=\i] in {1,missing,2,missing,3}
    \node [every neuron/.try, neuron \m/.try ] (output-\m) at (\i+2.22,4) {};

    \foreach \m [count=\i] in {1,2,missing,3,missing,4}
    \node [every neuron/.try, neuron \m/.try ] (hidden-\m) at (\i*1.5,2) {};

    \foreach \m [count=\i] in {1,missing,2,missing,3}
    \node [every neuron/.try, neuron \m/.try] (input-\m) at (\i+2.22,0) {};

    % ================================================================================================
    % 神经元标记
    % ================================================================================================
    \foreach \m [count=\i] in {1,j,l}
    \node [above] at (output-\i.north) {$y_\m$};

    \foreach \m [count=\i] in {1,2,h,q}
    \node [above=0.1,right] at (hidden-\i.east) {$b_\m$};

    \foreach \m [count=\i] in {1,i,d}
    \node [below] at (input-\i.south) {$x_\m$};


    % ================================================================================================
    % 注解
    % ================================================================================================
    \draw [red,dashed,->] (output-2) -- ++(0.8,-0.35) -- ++(3.4,0)
    node [below=0.33,right] {\shortstack{第 $j$ 个输出神经元的输入 \\ $\beta_j = \sum\limits_{h=1}^{q} w_{hj}b_h$}};

    \draw [red,dashed,->] (hidden-3) -- ++(0.8,-0.45) -- ++(2.6,0)
    node [below=0.33,right] {\shortstack{第 $h$ 个隐层神经元的输入 \\ $\alpha_j = \sum\limits_{i=1}^{d} v_{ih}x_i$}};

    % ================================================================================================
    % 连接
    % ================================================================================================
    \foreach \i in {1,...,4}
    \foreach \j in {1,3}
    \draw [-] (hidden-\i.north) -- (output-\j.south);

    \foreach \i in {1,...,4}
    \draw [red,-] (hidden-\i.north) -- (output-2.south);

    \foreach \i in {1,...,3}
    \foreach \j in {1,2,4}
    \draw [-] (input-\i.north) -- (hidden-\j.south);

    \foreach \i in {1,...,3}
    \draw [red,-] (input-\i.north) -- (hidden-3.south);

    % ================================================================================================
    % 标记
    % ================================================================================================
    \foreach \m [count=\i] in {1,2,h,q}
    \node [yshift=10,fill=white,text=red,scale=0.8] at (hidden-\i.north) {$w_{\m j}$};

    \foreach \m [count=\i] in {1,i,d}
    \node [yshift=10,fill=white,text=red,scale=0.8] at (input-\i.north) {$v_{\m j}$};

    % ================================================================================================
    % 注解
    % ================================================================================================
    \foreach \l [count=\i from 0] in {输入, 隐, 输出}
    \node [align=center, left] at (0,\i*2) {\l 层};

  \end{tikzpicture}
\end{center}

如图所示,该神经网络是一个拥有 $d$ 个输入神经元,$l$ 个输出神经元,$q$ 个隐层神经元的多层前馈网络结构。

其中输出层第 $j$ 个神经元的阈值用 $\theta_j$ 表示,隐层第 $h$ 个神经元的阈值用 $\gamma_h$ 表示;
输入层第 $i$ 个神经元与隐层第 $h$ 个神经元之间的连接权位 $v_{ih}$,隐层第 $h$ 个神经元与输出层第 $j$ 个神经元之间的连接权位 $w_{hj}$。

其中隐层的第 $h$ 个神经元接收到的输入(图中输入层与隐层之间的三条红线所示)为 $\alpha_h = \sum_{i=1}^{d} v_{ih} x_i$;
输出层的第 $j$ 个神经元接收到的输入(图中隐层与输出层之间的四条红线所示)为 $\beta_j = \sum_{h=1}^{q} w_{hj} b_h$,
其中 $b_h$ 为隐层第 $h$ 个神经元的输出。

对训练例 $(\pmb{x}_k, \pmb{y}_k)$ 假设神经网络的输出为 $\hat{\pmb{y}}_k = (\hat{y}_1^k,\hat{y}_2^k,\dots,\hat{y}_l^k,)$,即
\begin{equation}
  \hat{y}_j^k = f(\beta_j - \theta_j)
\end{equation}

这里的真实值 $y_j^k = (y_1^k,y_2^k,\dots,y_l^k)^2$,那么对于某第 i 个输出层的神经元而言,其均方误差即
\[
  \frac{(\hat{y}_i^k - y_i^k)}{2}
\]
那么将所有神经元加总,就有了网络在 $(\pmb{x}_k, \pmb{y}_k)$ 上的均方误差为:
\begin{equation}
  E_k = \frac{1}{2} \sum_{j=1}^{l} (\hat{y}_j^k - y_j^k)^2
\end{equation}

另外上图的网络中有 $(d + l + l) q + l$ 个参数需确定:输入层到隐层的 $d \times q$ 个权值、隐层到输出层的 $q \times l$ 个权值、
$q$ 个隐层神经元的阈值、$l$ 个输出层神经元的阈值。BP 是一个迭代学习算法,在迭代的每一轮中采用广义的感知机学习规则对参数进行更新估计,
即与式 5.1 类似,任意参数 $v$ 的更新估计式为
\begin{equation}
  v \leftarrow v + \Delta v.
\end{equation}

BP 算法基于 \textbf{梯度下降(gradient descent)} 策略,以目标的\textit{负}梯度方向\textit{对参数进行调整}。
对式 5.4 的误差 $E_k$,给定学习率 $\eta$,有
\begin{equation}
  \Delta w_{hj} = -\eta \frac{\partial E_k}{\partial w_{hj}}
\end{equation}

这里提到的负梯度,是利用了导数趋近于零时可以得到最小值的特性,去求得均方误差 $E_k$ 的最小值。回顾上文中的感知机学习算法:
\begin{aquote}
  ...随机选取一个误分类点并使其梯度下降,所以权重 $\pmb{w}$ 的更新式为
  \begin{gather*}
    \pmb{w} \leftarrow \pmb{w} + \Delta \pmb{w}, \\
    \Delta \pmb{w} = -\eta (\hat{y}_i - y_i) \pmb{w} = \eta (y_i - \hat{y}_i) \pmb{w}
  \end{gather*}
\end{aquote}

式 5.6 与感知机的最小化损失函数的不同之处在于:感知机是有两层神经元构成的,且输出层只有一个神经元,因此只需要考虑一层的权重变化,
即 $(\hat{y}_i - y_i) \pmb{w}$;而对于由若干个感知机所构成的神经网络而言,所有权重的变化则变为了
$\frac{\partial E_k}{\partial w_{hj}}$。

那么对于多层的神经网络,式 5.4 仅表现出了最后一层隐层与输出层的权重变化,那么就有了疑问:其它层之间的权重好像并不会被式 5.6 所改变?
接下来继续,注意到 $w_hj$ 先影响到第 $j$ 个输出层神经元的输入值 $\beta_j$。根据图示其为隐层所有神经元与其本身的表达式,即
\[ \beta_j = \sum_{h=1}^{q} w_{hj} b_h \]
接着 $w_{hj}$ 再影响到其他输出值 $\hat{y}_j^k$,最后再影响到 $E_k$,那么根据求导的链式法则有:
\begin{equation}
  \frac{\partial E_k}{\partial w_{hj}} =
  \frac{\partial E_k}{\partial \hat{y}_j^k} \cdot
  \frac{\partial \hat{y}_j^k}{\partial \beta_j} \cdot
  \frac{\partial \beta_j}{\partial w_{hj}}
\end{equation}

根据 $\beta_j$ 的定义,有:
\begin{equation}
  \begin{split}
    \frac{\partial \beta_j}{\partial w_{hj}} & = \frac{\partial (\sum_{h=1}^{q} w_{hj} b_h) }{\partial w_{hj}} \\
    & = \frac{\partial (w_{1j}b_1 + w_{2j}b_2 + \cdots + w_{hj}b_h + \cdots + w_{qj}b_q)}{\partial w_{hj}} \\
    & = 0 + 0 + \cdots + b_h + \cdots + 0 \\
    & = b_h
  \end{split}
\end{equation}

而又根据 Sigmoid 函数的一个很好的性质:
\begin{equation}
  f'(x) = f(x)(1 - f(x))
\end{equation}

根据式 5.4 与式 5.3 有:
\begin{equation}
  \begin{split}
    g_j & = -\frac{\partial E_k}{\partial \hat{y}_j^k} \cdot \frac{\partial \hat{y}_j^k}{\partial \beta_j} \\
    & = -(\hat{y}_j^k - y_j^k)f'(\beta_j - \theta_j) \\
    & = \hat{y}_j^k (1 - \hat{y}_j^k)(y_j^k - \hat{y}_j^k)
  \end{split}
\end{equation}

将式 5.10 与式 5.8 代入式 5.7,再代入式 5.6,就得到了 BP 算法中关于 $w_{hj}$ 的更新公式:
\begin{equation}
  \begin{split}
    \Delta w_{hj} & = -\eta \frac{\partial E_k}{\partial w_{hj}} \\
    & = -\eta \frac{\partial E_k}{\partial \hat{y}_j^k} \cdot
    \frac{\partial \hat{y}_j^k}{\partial \beta_j} \cdot
    \frac{\partial \beta_j}{\partial w_{hj}} \\
    & = \eta \hat{y}_j^k (1 - \hat{y}_j^k)(y_j^k - \hat{y}_j^k) \cdot b_h \\
    & = \eta g_j b_h
  \end{split}
\end{equation}

类似可得:
\begin{equation}
  \Delta \theta_j = -\eta g_j
\end{equation}
\begin{equation}
  \Delta v_{ih} = \eta e_h x_i
\end{equation}
\begin{equation}
  \Delta \gamma_h = -\eta e_h
\end{equation}

% TODO

\subsection{全局最小与局部最小}

WIP

\end{document}
