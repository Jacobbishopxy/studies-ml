% @file:		neural_networds.tex
% @author:	Jacob Xie
% @date:		2023/03/18 11:57:29 Saturday
% @brief:

\documentclass[../studies-ml.tex]{subfiles}

\begin{document}

\subsection{神经元模型}

\begin{center}
  \begin{tabular}{ |p{3cm}||p{4cm}|p{6cm}|  }
    \hline
    \multicolumn{3}{|c|}{\textbf{神经元模型}}                                               \\
    \hline
    名称   & 英文                  & 描述                                                    \\
    \hline
    神经网络 & neural networks     & 由具有适应性的简单单元组成的广泛并行互联的网络,它的组织能够模拟生物神经系统对真实世界物体所作出的交互反应 \\
    \hline
    神经元  & neuron              & 神经网络中最基本的成分                                           \\
    \hline
    阈值   & threshold           & 公式中记作 $\theta$                                        \\
    \hline
    连接   & connection          &                                                       \\
    \hline
    激活函数 & activation function & 处理以产生神经元的输出                                           \\
    \hline
    挤压函数 & squashing function  &                                                       \\
    \hline
  \end{tabular}
\end{center}

\subsection{感知机与多层网络}

感知机(Perceptron)由两层神经元组成。如图示,输入层接受外界输入信号后传递给输出层,输出层是 M-P 神经元,也称阈值逻辑单元
(threshold logic unit),其中 $y = f(\sum_i w_i x_i - \theta)$,而 $f$ 为激活函数。

\begin{center}
  \begin{tikzpicture}[shorten >=1pt,->]
    \tikzstyle{unit}=[draw,shape=circle,minimum size=1]

    \node[unit](p) at (0,1){};
    \node[unit](x1) at (-0.8,0){};
    \node[unit](x2) at (0.8,0){};
    \node[below] at (x1.south){$x_1$};
    \node[below] at (x2.south){$x_2$};
    \node[below,left=0.35,yshift=-8] at (p.south){$w_1$};
    \node[below,right=0.35,yshift=-8] at (p.south){$w_2$};

    \draw [->] (x1) -- (p);
    \draw [->] (x2) -- (p);
    \draw (p) -- ++(0,1) node[xshift=-10]{$y$};

    \node [align=center, right=1.6] at (p) {\footnotesize{输出层}};
    \node [align=center, right=0.8] at (x2) {\footnotesize{输入层}};
  \end{tikzpicture}
\end{center}

一般而言,给定训练数据集,权重 $w_i\, (i=1,2,\dots,n)$ 以及阈值 $\theta$ 可通过学习得到。而阈值 $\theta$ 可视为一个固定输入
为 -1.0 的哑结点(dummy node)所对应的链接权重 $w_{n+1}$,因此权重和阈值的学习就可以统一为权重的学习。感知机的学习规则非常简单,
对训练样例 $(\pmb{x},y)$ 而言,如果当前感知机的输出位 $\hat{y}$,则感知机权重将这样调整:

\begin{equation}
  w_i \leftarrow w_i + \Delta w_i
\end{equation}

\begin{equation}
  \Delta w_i = \eta (y - \hat{y}) x_i
\end{equation}

其中 $\eta \in (0,1)$ 称为学习率(learning rate)。公式 5.2 为感知机学习算法中的参数更新式。

\begin{enumerate}[I]
  \item \textbf{感知机模型}

        感知机模型的式可表示为:

        \begin{align*}
          \begin{split}
            y & = f \Biggl( \sum_{i=1}^{n} w_i x_i - \theta \Biggr) \\
            & = f(\pmb{w}^T \pmb{x} - \theta)
          \end{split}
        \end{align*}

        其中,$x \in \mathbb{R}^n$ 即样本的特征向量,是感知机模型的输入;$\pmb{w}, \theta$ 是感知机模型的参数,权重 $w \in \mathbb{R}^n$,
        $\theta$ 为阈值。假定 $f$ 为阶跃函数,那么感知机模型的式可以表示为:

        \begin{equation*}
          y = \varepsilon (\pmb{w}^T \pmb{x} - \theta) = \begin{cases}
            1,\quad \pmb{w}^T \pmb{x} - \theta \ge 0; \\
            0,\quad \pmb{w}^T \pmb{x} - \theta < 0.
          \end{cases}
        \end{equation*}

        由于 $n$ 维空间中的超平面方程为
        \[
          w_1 x_1 + w_2 x_2 + \cdots + w_n x_n + b = \pmb{w}^T \pmb{x} + b = 0
        \]

        因此感知机模型式中的 $\pmb{w}^T \pmb{x} - \theta$ 可视为 $n$ 维空间中的一个超平面,将 $n$ 维空间划分为 $\pmb{w}^T \pmb{x} - \theta \ge 0$
        与 $\pmb{w}^T \pmb{x} - \theta < 0$ 的两个子空间(\textit{试想一下三维空间下的一个平面将空间切分为两部分})。
        那么落在前一个子空间的样本对应的模型输出值为 1,而后者为 0,如此实现了分类功能。

  \item \textbf{学习策略}

        给定一个线性可分的数据集 $T$,感知机的学习目标是求得能对数据集 $T$ 中的正负样本完全正确划分的分离超平面
        \[ \pmb{w}^T \pmb{x} - \theta = 0 \]

        假设此时误分类样本集合为 $M \subset T$,对任意一个误分类样本 $(\pmb{x},y) \in M$ 而言,当 $\pmb{w}^T \pmb{x} - \theta \ge 0$ 时,
        模型输出值为 $\hat{y} = 1$,样本真实标记为 $y = 0$;反之亦然。综上,以下式恒成立:
        \[ (\hat{y} - y)(\pmb{w}^T \pmb{x} - \theta) \ge 0 \]

        因此对于给定数据集 $T$,其损失函数可以定义为
        \[ L(\pmb{w}, \theta) = \sum_{\pmb{x} \in M} (\hat{y} - y)(\pmb{w}^T \pmb{x} - \theta) \]

        非负之和显然非负,因此损失函数为非负。当没有误分类点时,损失函数的值为 $0$;误分类点越少,误分类点离超平面越近,损失函数值就越小。
        因此对于给定的数据集 $T$,损失函数 $L(\pmb{w}, \theta)$ 是关于 $\pmb{w}, \theta$ 的连续可导函数(注意是关于 $\pmb{w}, \theta$
        的可导,意味着之后将要对其进行梯度下降算法,即对其使用导数计算)。

        连续:
        \[
          \lim_{x \to x_0} f(x) = f(x_0)
        \]

        可导:
        \[
          f'(x_0) = \lim_{\Delta x \to 0} \frac{\Delta y}{\Delta x} =
          \lim_{\Delta x \to 0} \frac{f(x_0 + \Delta x) - f(x_0)}{\Delta x}
        \]

  \item \textbf{学习算法}

        感知机模型的学习问题可以转化为求解损失函数的最优化问题。给定数据集
        \[ T = \{(\pmb{x}_1,y_1),(\pmb{x}_2,y_2),\cdots,(\pmb{x}_N,y_N),\} \]
        其中 $\pmb{x}_i \in \mathbb{R}, y_i \in \{0,1\}$,求参数 $\pmb{w}, \theta$ 使得损失函数最小化:
        \[
          \min_{\pmb{w},\theta} L(\pmb{w},\theta) =
          \min_{\pmb{w},\theta} \sum_{\pmb{x}_i \in M} (\hat{y}_i - y_i)(\pmb{w}^T \pmb{x} - \theta)
        \]

        其中 $M \subset T$ 为误分类样本集合。若将阈值 $\theta$ 视为一个固定输入为 $-1$ 的 “哑结点”(前文有提到),即:
        \[ -\theta = -1 \cdot w_{n+1} = x_{n+1} \cdot w_{n+1} \]

        那么 $\pmb{w}^T \pmb{x} - \theta$ 可简化为

        \begin{align*}
          \begin{split}
            \pmb{w}^T \pmb{x} - \theta & = \sum_{j=1}^{n} w_j x_j + x_{n+1} \cdot w_{n+1} \\
            &= \sum_{j=1}^{n+1} w_j x_j \\
            &= \pmb{w}^T \pmb{x_i}
          \end{split}
        \end{align*}

        其中 $\pmb{x_i} \in \mathbb{R}^{n+1},\, \pmb{w} \in \mathbb{R}^{n+1}$,有此可将最小化问题进一步简化:
        \[
          \min_{\pmb{w}} L(\pmb{w}) = \min_{\pmb{w}} \sum_{\pmb{x_i} \in M} (\hat{y}_i - y_i) \pmb{w}^T \pmb{x_i}
        \]

        假设误分类样本集合 $M$ 固定,那么可以求得损失函数 $L(\pmb{w})$ 的梯度
        \[ \nabla_{\pmb{w}} L(\pmb{w}) = \sum_{\pmb{x_i} \in M} (\hat{y}_i - y_i) \pmb{x_i} \]

        感知机的学习算法具体采用的是随机梯度下降法,即在最小化的过程中,不是一次使 $M$ 中所有误分类点的梯度下降,而是一次随机选取
        一个误分类点,并使其梯度下降。所以权重 $\pmb{w}$ 的更新式为
        \begin{gather*}
          \pmb{w} \leftarrow \pmb{w} + \Delta \pmb{w}, \\
          \delta \pmb{w} = -\eta (\hat{y}_i - y_i) \pmb{w} = \eta (y_i - \hat{y}_i) \pmb{w}
        \end{gather*}

        即 $\pmb{w}$ 中的某个分量 $w_i$ 的更新式即式 5.2。
\end{enumerate}

\begin{center}
  \begin{tabular}{ |p{3cm}||p{4cm}|p{6cm}|  }
    \hline
    \multicolumn{3}{|c|}{\textbf{神经元模型}}                    \\
    \hline
    名称       & 英文                                      & 描述 \\
    \hline
    功能神经元    & functional neuron                       &    \\
    \hline
    线性可分     & linearly separable                      &    \\
    \hline
    收敛       & converge                                &    \\
    \hline
    震荡       & fluctuation                             &    \\
    \hline
    隐层或隐含层   & hidden layer                            &    \\
    \hline
    多层前馈神经网络 & multi-layer feedforward neural networks &    \\
    \hline
    连接权      & connection weight                       &    \\
    \hline
  \end{tabular}
\end{center}


\subsection{误差逆传播算法}

%

\end{document}
